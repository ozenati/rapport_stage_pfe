\documentclass{beamer}
\usetheme{Warsaw}
\useoutertheme{infolines}
\setbeamertemplate{headline}[default]
%\setbeamercovered{transparent}
%\setbeamertemplate{footline}[default]
%\setbeamertemplate{headline}[split]
\usepackage[T1]{fontenc}
\usepackage[utf8x]{inputenc}
\usepackage[english]{babel}
\usepackage{color}  
%%%%%%%%%definition des variables%%%%%%%%%%%%%%%
%\usepackage[latin1]{inputenc}
%\usepackage[T1]{fontenc}
%\usepackage{textcomp}
%\usepackage{lmodern}
%\usepackage{listings} 


%%%%%%%%%%%%%%%%%%%%%%%%%%%%%
\def\sujet{}
%Oral Presentation of Parallel Architecture
\def\projet{Efficient Address Mapping of Shared Cache for On-Chip Many-Core Architecture} 
\def\etape{}
\def\gA{Omar \textsc{Zenati}}
\def\prof{Susan \textsc{Medina} \& Denis \textsc{Barthou}}


%%%%%%%%%%%%%%%% Header %%%%%%%%%%%%%%%%
 \title[Oral Presentation]{
        {\bfseries \projet\\} 
        {\bfseries \huge \sujet}
}

\institute[3A]{
  {\normalsize \bfseries \sffamily Professor:} {\large \prof}~~~\\ 
  
}

\author[Zenati]{
  {\normalsize \bfseries \sffamily Student:} {\large \gA}~~~~~~~~~~\\
}

\begin{document}
        
\begin{frame}
\maketitle
\end{frame}

\begin{frame}{Plan}
\tableofcontents
\end{frame}

\AtBeginSection[]{
  \begin{frame}{Summary}
  \tableofcontents[currentsection,hideallsubsections]
  \end{frame}
}

\section{Introduction}
\begin{frame}
\frametitle{Introduction}
\framesubtitle{Moore's Law}
Moore states: \textit{"Computer chips double in speed every 1.5 to 2 years period"}
\pause
\begin{exampleblock}{}
\underline{Reason:}
\begin{center}
Use of smaller transistors
\end{center}
\end{exampleblock}{}
\pause
\underline{Problem:} Stopped by Heat Wall
\pause
\begin{exampleblock}{}
\underline{Solution:}
\begin{center}
Increase of number of cores
\end{center}
\end{exampleblock}{}
\pause
\underline{Problem:} Stopped by Memory Wall
\pause
\begin{exampleblock}{}
\underline{Solution:} 
\begin{center}
Optimise at all levels
\end{center}
\end{exampleblock}{}
\end{frame}

\begin{frame}
\frametitle{Introduction}
\framesubtitle{The cache}
\begin{figure}
\includegraphics[width=0.8\textwidth]{archi_multiprocessor.jpg}
\end{figure}
\begin{center}
Access to the cache is faster than access memory !
\end{center}
\end{frame}



\section{XORM Address Mapping Scheme}
\begin{frame}
\frametitle{Cache}
\framesubtitle{Organisation}
\begin{exampleblock}{}
\begin{itemize}
\item \underline{Cache Line:} Data bloc corresponding to a contiguous data bloc in memory.
\item Cache lines are organized into sets of lines.
\end{itemize}
\end{exampleblock}{}
Each part of a memory address shows how to find the equivalent bloc in the cache:
\begin{exampleblock}{}
\begin{itemize}
\item Tag: Identification of the address
\item Cache set: Number of the equivalent set
\item Offset: Position in the cache line
\end{itemize}
\end{exampleblock}{}
\begin{figure}
\includegraphics[width=0.8\textwidth]{address_memory.png}
\end{figure}
\end{frame}

\begin{frame}
\frametitle{Cache}
\framesubtitle{Problem}
The conflict miss problem :
\begin{figure}
\includegraphics[width=0.8\textwidth]{problem.png}
\end{figure}
\end{frame}

\begin{frame}
\frametitle{Conflict Miss}
\framesubtitle{Some solution}
To reduce conflict miss in cache :
\begin{exampleblock}{}
\begin{itemize}
\item Skewed-associative cache
\item Victim cache
\item Column-associative cache
\item Prime-indexed cache
\item ...
\end{itemize}
\end{exampleblock}{}
\pause
The main shortcomings of these schemes are:
\begin{exampleblock}{}
\begin{itemize}
\item Computing complexity 
\item Implementing cost
\end{itemize}
\end{exampleblock}{}
\end{frame}


\begin{frame}
\frametitle{XORM scheme}
\framesubtitle{Principle}
\underline{Definition :}
\begin{figure}
\includegraphics[width=0.8\textwidth]{def.png}
\end{figure}
\pause
\begin{figure}
\includegraphics[width=0.5\textwidth]{xorm_scheme.png}
\end{figure}
\pause
\begin{exampleblock}{}
\begin{center}
$ index = A_3 H_{m*m} \oplus A_2$\\
$ interleave = A_4 H_{m*m} \oplus A_1$
\end{center}
\end{exampleblock}{}
\end{frame}


\section{Simulation and Result}
\begin{frame}
\frametitle{Simulation}
\framesubtitle{Godson-T Simulator Architecture}
\begin{figure}
\includegraphics[width=0.8\textwidth]{godsont.png}
\end{figure}
\end{frame}

\begin{frame}
\frametitle{Simulation}
\framesubtitle{Target algorithm}
The simulation was applied with :
\begin{exampleblock}{}
\begin{itemize}
\item Matrix multiplication
\item FFT decomposition
\item LU decomposition
\item Pfind algorithm
\end{itemize}
\end{exampleblock}{}
\end{frame}

\begin{frame}
\frametitle{Results}
\begin{figure}
\includegraphics[width=0.8\textwidth]{result1.png}
\end{figure}
\end{frame}

\begin{frame}
\frametitle{Results}
\begin{figure}
\includegraphics[width=0.8\textwidth]{result2.png}
\end{figure}
\end{frame}


\section{Conclusion}
\begin{frame}
\frametitle{Conclusion}
\underline{XORM scheme :}
\begin{exampleblock}{}
\begin{itemize}
\item Less miss conflict
\item Better performance
\end{itemize}
\end{exampleblock}{}
\underline{Questions :}
\begin{exampleblock}{}
\begin{itemize}
\item What about others architectures?
\end{itemize}
\end{exampleblock}{}
\end{frame}


\frame{
\frametitle{Conclusion}
\begin{exampleblock}{}
\begin{center}
\huge{Thank you for your attention}
\end{center}
\end{exampleblock}{}
}

\end{document}
