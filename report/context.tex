The work presented in this report was carried out under an internship. This was performed during three months in the Innovative Computer Laboratory\footnote{\url{http://icl.cs.utk.edu/}} (ICL) and three months in the Inria of Bordeaux. It is part of the project \emph{Matrices Over Runtime Systems @ Exascale}\footnote{\url{http://icl.eecs.utk.edu/morse/}} (MORSE) which aims to design dense and sparse linear algebra methods that achieve the fastest possible time to an accurate solution on large-scale multi-core systems with GPU accelerators.

\subsection*{Innovative Computer Laboratory}
Attached to the university of Tennessee, the Innovative Computer Laboratory (ICL) is one of the world leader laboratory in the field of high performance computing (HPC). ICL was established by the Pr Jack Dongarra from 1989. Since its inception, ICL produced and participated to several applications of high value to the HPC community including: ATLAS, BLAS, LAPACK, MPI, Netlib, PAPI, ScaLAPACK, Top 500 \dots

Today, ICL continues in its desire to contribute to science. In this dynamic, I worked with the Distributed Computing Group team where my mission was to apprehend the \dague runtime system first and then to study the feasibility of task flow LU decomposition over PTG using \dague as practical tool.

\subsection*{Inria}
%Public science and technology institution established in 1967, Inria is is the only public research body fully dedicated to computational sciences. Combining computer sciences with mathematics, Inria’s 3,400 researchers strive to invent the digital technologies of the future.
The National Institute for Research in Computer Science and Control (INRIA) is a public science and technology institution. It was created in 1967. It includes 8 research centers in France and has over 4000 employees. INRIA depends on the French ministries of research and industry.

The institute has strong relationships internationally. It has various partnerships in Africa, Middle East, America and Asia. Thus, it promotes exchanges between scientists around the world.

I worked with the BACCHUS team which aim to develop and validate numerical methods adapted to physical problems. Firstly, my goal was to continue the work I started at ICL. Then, my mission was to take in hand the StarPU runtime system and implement its task flow LU decomposition in order to compare performances of PTG and sequential task flow.