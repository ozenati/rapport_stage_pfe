% More heterogeneity calls for reactive programming model 

Numerical simulation has become so pervasive that it is often considered
as the fourth pillar of science. Indeed, this insatiable need of
computing power has fuelled the production of ever more powerful, larger
High Performance Computing (HPC) systems. In recent years, power
consumption and thermal issues have stalled the decades old trend of
performance boost from clock frequency increase. Consequently, modern
designs, such as the Titan supercomputer currently being built at the
Oak Ridge National Laboratory, have to resort to packaging heterogeneous
core types (including GPU accelerators), in larger number and with
increasingly intricate Non Uniform Memory Access (NUMA) hierarchies.
These disruptive technological trends take a toll on productivity:
heterogeneity and essentially unpredictable time for memory accesses concur
to jitter and asynchrony, which are challenging established programming
models designed around SPMD, homogeneous and regular computation.

% dataflow is the answer, dague is the tool

This \emph{jungle}\footnote{Herb Sutter, ``Welcome to the Jungle'',
12-29-2011,
\url{http://herbsutter.com/2011/12/29/welcome-to-the-jungle/}} of
resources has called forth the emergence of novel programming languages
and tools that enable a more adaptive reaction to unexpected contentions
and delays, predominantly by expressing parallelism as a
task flow ~\cite{springerlink:10.1007/978-3-642-19328-6_5,PBL08,FYD09,DAGuE:PARCO,ATNW11,charm}.
In this execution model, the program is broken into elementary tasks and
a Direct Acyclic Graph (DAG) represents the dependency flow imposed by
data sharing between tasks. At runtime, this DAG can be scheduled with a
dynamic task distribution according to on-the-spot resource
introspection and work stealing in an architecture-aware fashion, so as
to overlap communications, tolerate jitter, and benefit from performance
portability by hiding the complexities inherent to heterogeneous
hardware. Among task flow approaches, the \dague tool kit stands out by
using a Parametrized Task Graph (PTG) representation~\cite{CJY04} of the
DAG. This compact and symbolic representation is cornerstone to the
scalability enjoyed by the \dague version of dense linear
algebra routines~\cite{DAGuE:PARCO}, as it enables an independent,
distributed runtime evaluation of the successor and predecessors of any
task, at any moment, without the need for unfolding the entire DAG.

% algebraic representation is static, GEPP is dynamic

Many applications use the LU factorization algorithm to solve
linear systems of equations. This popularity roots in two facets: the
computational complexity of LU is half that of QR, yet its numerical
accuracy is similarly excellent, thanks to a mechanism called partial
pivoting. With partial pivoting, the best, most stable pivot for the LU
reduction is selected in the entire column and lines are swapped
accordingly to promote this pivot to the diagonal. The pivot
selection cannot be pre-computed, therefore
% its outcome for a particular iteration
%depends on the current content of the matrix after previous updates have
%been applied. Considering a taskflow task, the lines it depends on is
%conditional on the pivot selection and according swapping, and cannot be
%inferred before actual computation takes place: 
the DAG representing LU with partial pivoting changes depending on the
updated content of the matrix. Yet, the PTG representation used in
\dague can be generated from a compile time analysis, and imposes a static
structure to the DAG itself. The data-dependent nature of the DAG
representing LU with partial pivoting is thereby difficult to capture in
a PTG representation without increasing tremendously the number of tasks
and edges to be considered at runtime, a significant source of overhead
and memory waste. The main contribution of this paper is to introduce a
variation of the LU algorithm with partial pivoting that can be
expressed in a PTG representation efficiently.

%, thereby enabling an evaluation of a stable,
%numerically accurate version of the LU factorization in such a taskflow
%system.

% outlines

The rest of this paper is organized as follows. Section~\ref{background}
presents related work in the field of task flow scheduling and
implementations of the LU algorithm. Then, sections ~\ref{panel} and ~\ref{update} further
discuss the issues raised by partial pivoting in PTG task flow systems, and
the solutions we propose to tackle these challenges.
Section~\ref{experimental} presents an experimental evaluation of the
resulting algorithm and compares it with both a legacy SPMD
implementation of LU with partial pivoting and the less stable but less
challenging LU with incremental pivoting in a similar task flow
implementation, before we conclude.
